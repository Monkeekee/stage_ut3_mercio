% !TeX root = rapport_stage.tex
\documentclass{rapportCS}
\title{Rapport UT3 - Mercio} %Thanks for Rapport CentraleSupelec - Template, By Axel Poupart-Lafarge
\begin{document}

%----------- Informations du rapport ---------

\logoentreprise{images/Black_Full_Small.png}

\titre{Rapport de stage} % Titre du fichier
\sujet{ Alignement concurrentiel automatique } % Sujet du stage

\mention{Mention Données et Connaissances } % Nom de la Mention
\trigrammemention{M2-DC} % Pour le bas de la page
\master{Master d'Informatique} % Nom du master
\filiere{Promotion 2021-2022} % Nom de la filière

\eleve{Matthieu Akhavan}

\dates{11/04/2022 - 30/09/2022}

% Informations tuteurs écoles
\tuteurecole{
    Mention : \textsc{Chloé BRAUD} \\
    chloe.braud@irit.fr \\
} 

\tuteurentreprise{
    \textsc{Guillaume HALB} \\
    guillaume.halb@mercio.io \\
		\textsc{Yoan OZANNEAUX} \\
    yoann.ozanneaux@mercio.io
}

%----------- Initialisation -------------------
        
\fairemarges %Afficher les marges
\fairepagedegarde %Creer la page de garde

%----------- Abstract -------------------
\vspace*{\stretch{1}}
\begin{center}
	\begin{abstract}
A COMPLETER + PAREIL EN ANGLAIS, 500 mots \\

Ceci est le rapport de mon stage de 5 mois et deux semaines dans l'entreprise Mercio, à Paris,
spécialisé dans le retail et la Big Data.
C'est jusqu'à ce jour l'expérience professionnelle la plus complète que j'ai eue au cours de mes 
études en informatique. \\
Mon sujet tourne autour d'une problématique qui allie des enjeux données et intelligence 
artificielle, mais aussi développement logiciel pour l'implémentation de la solution.
    \end{abstract}
\end{center}
\vspace*{\stretch{1}}
\newpage

%------------ Table des matières ----------------

\tabledematieres % Créer la table de matières

%------------ Corps du rapport ----------------


%------------ Introduction ----------------

\section{Mise en contexte} 

\subsection{Introduction générale}
	Mercio, de raison sociale HereForRetail, est une société dans le secteur du retail. 
	C'est à dire que ses clients sont les maillons finaux de la chaîne de la grande distribution 
	française, qui vendent des produits en rayons, que ce soit dans l'alimentaire, le bricolage, 
	ou les jouets. Mercio propose à ses clients un logiciel web qui leur permet d'ajuster leurs prix 
	en rayon selon les stratégies définies en interne. Les stratégies s'expriment alors sous forme 
	d'un ensemble de règles de prix configurables sur l'application Mercio Pricing, dont toutes les 
	données sont emmagasinées dans le nuage. Les règles que ces clients peuvent appliquer à leurs 
	produits ont des priorités entre elles selon la stratégie de l'enseigne et sont paramétrables en 
	profondeur : il y a la très connue règle d'arrondi (prix qui fini par 99 centimes par exemple), 
	l'alignement par rapport à la concurrence, et bien d'autres. \\
	Avec les milliers de magasins que certains clients ont sur le territoire français et les milliers 
	de produits vendus en magasins, le nombre de prix à traiter pour les plus gros clients va jusqu'à 
	plusieurs dizaines de millions de prix. Le fait que la plateforme est capable de traiter autant de 
	prix en peu de temps (une dizaine de minutes la plupart du temps) place Mercio en qualité 
	d'entreprise Big Data dans le vaste secteur de la tech.\\
	Au fil des avancées dans le domaine lucratif des sciences des données, de plus en plus 
	d'entreprises proclament utiliser des techniques d'intelligence artificielle pour obtenir des 
	performances calculatoires jamais égalées par le passé, et ce avec des technologies de plus en plus
	à la pointe dans leur domaine. 
	Mercio n'est pas en reste, et c'est ainsi que j'ai répondu à leur appel
	d'offre pour résoudre un challenge de data science en tant que développeur stagiaire en fin de 
	Bac+5 informatique.\\
	Mon sujet au sein de l'entreprise est de développer une solution algorithmique et logicielle pour 
	faire une suggestion de correspondance entre chaque produit donné d'une enseigne client et le 
	produit jugé le plus similaire possible chez chaque concurrent 
	dont le client met à disposition les données. 
	Ce sujet, ses enjeux pour l'entreprise mais aussi pour moi en tant qu'apprenti,
	seront bien plus développés dans les parties suivantes.\\
	Cependant, avant de parler en profondeur du sujet, prendre connaissances des descriptions qui
	suivent sont capitales pour comprendre la place que j'ai en tant que stagiaire dans cette 
	entreprise de 21 employés, stagiaires compris (au moment ou je rédige cette introduction) car la
	dimension économique mais également le plan technologique dans lesquels se situe Mercio influencent 
	grandement sa structure et ses cycles de production. Tout autant sa manière de fonctionner que les 
	solutions technologiques choisies, rien n'est laissé au hasard, mais bel et bien conçu pour 
	répondre à des besoins fonctionnels spécifiques et à des moments clés du cycle de vie de 
	l'entreprise.

\subsection{Modèle économique de l'entreprise}
	La licence du logiciel de pricing est vendue à l'année et tous les différents modules de 
	l'application ne sont pas fournis à tous les clients, et de façon différentes.
	En effet, certains clients ne sont intéressés que par certains modules.
	Ces clients peuvent aussi négocier des modifications, voire fonctionnalités supplémentaires 
	qui sont facturées en fonction de la masse de travail requise pour ce faire. 
	Ce mode de fonctionnement est appelé B2B pour « Business to Business » en opposition 
	au B2C « Business to Customer ». \\
	Les développeurs, qui représentent le plus gros des employés de Mercio, sont scindés en 
	deux équipes : l'équipe projet qui traite avec les clients directement et appliquent des 
	correctifs et changements selon leurs besoins, et l'équipe produit qui développe les 
	fonctionnalités standard sur le long-terme. Je travaille actuellement sur mon sujet dans 
	l'équipe produit, qui s'apparente d'avantage au secteur recherche et développement de beaucoup 
	d'entreprises. Nous reviendrons dessus plus tard. \\
	Les enseignes de retail avec qui échangent les commerciaux de l'entreprise ne sont pas 
	exclusivement des enseignes françaises mais aussi européennes, notamment en Allemagne.
	Mercio Pricing est reconnu par beaucoup d'acteurs économiques clés du retail comme une solide 
	solution de pricing. 
	Le pricing est le fait d'attribuer un prix à un produit selon ses besoins stratégiques de vente. 
	Cette solution logicielle se positionne bien sur le marché en terme de qualité de recommandation 
	de prix et de quantité de données traitées par jour, avec des atouts stratégiques comme le 
	géo-pricing, une méthode de pricing qui prend en compte la concurrence locale de chaque magasin de 
	l'enseigne cliente de Mercio. \\


\subsection{Structure de l'application}
	L 'application Mercio Pricing est une solution web, accessible par les clients dans les horaires 
	prévues à cet effet en semaine. La solution d'un client est déployée sur un serveur distant et 
	accessible par identifiants et mots de passe sécurisés.\\
	Une fois connecté, le client a accès au service de pricing grâce aux données qu'il fourni sur son 
	site de stockage Microsoft Azure dédié.\\
	Pour synthétiser, l'application qu'utilisent les clients est divisée en deux partie : 
	la première est la solution standard qui est le cœur de l'application, nommée activepricing ou 
	appelé la plupart du temps le produit. Les travaux autour du produit rassemblent le gros des avancées 
	qui sont faîtes pour améliorer le service accordé à tous les clients de manière générale.\\
	La seconde partie est le starter, qui comme son nom l'indique, sert à démarrer un projet par 
	dessus le produit. Il est la partie personnalisable par les projets et doit être mise à jour suivant
	les changements majeurs apportés par les nouveautés du produit. Il arrive parfois que des 
	changements du produit apportent des inconvénients au niveau de projets pour des raisons techniques
	ou simplement par la spécifications des données des clients, mais je reviendrai dessus plus tard.\\
	Le produit consiste en une vue codé en React Typescript et trois unités interdépendantes codées
	en Java Spring Boot : le serveur, l'édition et le pricer. Je ne parlerai que de l'édition  coté
	serveur et de la vue car la partie développement logiciel de mon stage ne touche qu'à 
	ces composantes.\\

\subsection{ L'équipe produit }
	Les experts des prix qui utilisent la plateforme de pricing de Mercio chez les clients sont appelés 
	pricers. Fournir aux pricers des outils adaptés à leurs besoins est le nerf de la guerre pour 
	Mercio, qui cherche à gagner des nouveaux clients européens tout en standardisant de plus 
	en plus ses nouveaux contrats tout comme ses anciens. 
	L'objectif est de réduire le plus possible le temps passé à répondre à des besoins spécifiques 
	à travers les progrès réalisés sur le produit.\\
	L'équipe produit est composé de 7 personnes incluant :\\
	
	\begin{itemize}
	\item{Le manager de l'équipe produit Yoann Ozanneaux et tuteur officiel pour les stages.}
	\item{Mon tuteur d'entreprise Guillaume Halb, responsable de mes travaux au sein de l'équipe.}
	\item{Des développeurs qui ont de plusieurs mois à 4 ans d'ancienneté.}
	\item{Les deux stagiaires dont je fais partie.}
	\end{itemize}

\subsection{L'équipe projet}
	L'équipe projet s'occupe des solutions mises en œuvre pour les différents clients.
	Chaque projet ayant ses propres spécificités, ils sont attribués à un ou plusieurs membres
	de l'équipe en fonction de la nature et de la charge que constituent les demandes du 
	client pour des améliorations et du support. Un projet doit aussi participer à la formation
	du client à l'utilisation de la plateforme de pricing, anticiper ses demandes, faire respecter
	les exigences sur les données que le client fourni à l'entreprise, pré-traiter certaines données,
	etc.

\newpage

\section{Sujet du stage : Alignement horizontal de produits concurrents}

\subsection{Contexte de l'application}
Dans la plateforme ActivePricing, il y a de nombreuses façon d'appliquer des règles de prix sur les 
produits. On peut citer le pricing sur la valeur (ou value based pricing), la cohérence de prix,
ou bien alors l'alignement à la concurrence. Cette derinière est celle qui concerne le sujet qui suit.\\

L'alignement d'un produit A d'un client à sa concurrence signifie que pour chaque produit vendu par un
concurrent pour lequel on a déclaré une correspondance, on applique un prix qui suit un coefficient
qualitatif appliqué à ce triplet : Produit A, Enseigne concurrente E, produit concurrent X.
Concrètement, les pricers relient les produits de chez eux à des produits de leurs concurrents 
pour pouvoir appliquer un coefficient de prix et s'ajuster à la concurrence de manière qualitative,
quantitative ou arbitrairement une combinaison des deux. \\

Ce type de chainage entre produit est appelé 'horizontal' pour le différencier d'un autre type de 
chainage de produits, qui lui, se fait en interne sur les produits des clients uniquement, et qui est
géré par le module de cohérence. Qui plus est, on peut éventuellement combiner les deux : chainage
horizontal d'un produit A avec ceux de la concurrence et aussi avec un autre produit B vendu dans son 
propre magasin. \\

\subsection{Besoins de l'utilisateur final}
Quand le pricer utilise pour la première fois le module du "chainage horizontal",
les liens peuvent être très longs à tous écrire à la main s'il ne sont pas déjà exportés d'un fichier
rempli en interne. Il y a aussi le cas où on n'a aucun lien à exporter dans le module, et tout est à
faire en partant de zéro, en rentrant les références produits une par une en s'aidant de filtres.\\

L'idée derrière les recommandations automatiques de lien, ou pour faire court le matching automatique,
c'est que le pricer puisse rapidement appréhender ce module pour qu'il puisse efficacement utiliser
l'alignement à la concurrence le plus tôt possible.\\
D'un point de vue fonctionnalité, l'utilisateur aurait des liens à valider, 
avec plusieurs options pour filtrer selon les catégories de produits, afin de créer de bons liens
rapidement, sans avoir à taper et à chercher parmi tous les produits de la concurrence.\\

Bien entendu, le but derrière l'utilisation de la règle d'alignement à la concurrence est de faire un
maximum de profit en vendant ces produits similaires ou identiques que les concurrents
proposent aussi. \\
Si un distributeur est trop cher sur des produits clés par rapport à ses concurrents directs,
les consommateurs vont se diriger chez ceux-ci pour acheter ces produits clés. 
D'une autre manière, si un certain produit n'a pas d'équivallent chez ses concurrents directs,
alors on peut se permettre de garder les prix de vente élevés afin de garantir les marges faciles. \\
Les liens de chainage horizontal sont en partie une solution data à ce besoin en alignement.

\subsection{Répondre à la problématique et au besoin}
Mon travail consiste donc à concevoir un algorithme qui prend en entrée des produits concurrents et 
des produits clients pour calculer des liens cohérents en sortie pour tous les produits clients.
Ensuite, il s'agira d'implémenter ces résultats en tant que recommandations de liens dans la
plateforme Mercio Pricing.\\

\subsection{Données initiales}

\newpage

\section{Démarche proposée}
Concevoir une démarche est l'étape primordiale de tout sujet de recherche et développement.
Nous cherchons à savoir quels sont les connaissances à acquérir afin d'appliquer une solution adaptée,
quels sont les solutions similaires proposées dans la littérature, quels moyens possiède-t-on pour
déployer pragmatiquement la solution mise en oeuvre.\\

\subsection{Lien d'alignement concurrentiel : définition}
Pour pouvoir définir ce qu'est un bon lien d'alignement concurrentiel (ou simplement lien de matching),
il faut au prélable exposer quelques exemples concrets définis à la main pour se rendre compte de la 
difficulté de la conception d'un tel algorithme. \\

% exemple de cas facile mais pas le meme produit, 2 MDD, mano
INSERER IMAGE EXEMPLE MATCHING 1\\

Ici, un pricer décide de lier un produit A de son enseigne au produit X de son concurrent. 
Le produit A et le produit X possèdent plusieurs caractéristiques comme un nom, des catégories, un volume,
un paquetage, une marque, un prix, et bien d'autres dans la réalité.\\
Nous sommes ici dans le cas facile où le produit n'est pas le exactement identique à celui de son concurrent,
mais le produit semble concorder aux attentes de l'allignement concurrentiel : il s'agit de la marque distributeur
de l'enseigne qui vend ces produit dans les deux cas,
% note de base de page pour MDD et MN
les noms semblent sémantiquement désigner le meme produit, l'odre de grandeur du volume vendu en rayon est proche,
les catégories concordent assez bien, etc.
Nous avons tout pour dire qu'il est cohérent d'aligner A avec X pour ce concurrent-ci. \\

% exemple de cas qui foire à cause de tout sauf du nom, généré
INSERER IMAGE EXEMPLE MATCHING 2\\



\end{document}
